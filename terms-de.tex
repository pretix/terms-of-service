\documentclass{terms}
\usepackage{multicol}
\usepackage{enumitem}
\usepackage{scrextend}
\usepackage[utf8x]{inputenc}
\usepackage[ngerman]{babel}

\SetShortTitle{AGB für Verträge über die Ticketshop-Software pretix}
\SetTitle{Allgemeine Geschäftsbedingungen\\ für Verträge über die Ticketshop-Software pretix}
\SetDate{26.01.2021}
\SetVersion{1.4}

\hypersetup{
    pdftitle={\Title},
    pdfauthor={pretix.eu},
    pdfsubject={},
    pdfkeywords={},
}
\setcounter{tocdepth}{1}
\begin{document}
\maketitle

\begin{center}
rami.io GmbH\\
Markgräfler Straße 16\\
69126 Heidelberg
\end{center}
\sloppy

\section{Geschäftsbedingungen}
\subsection{Allgemeines}
Diese AGB gelten für alle Verträge zwischen der rami.io GmbH, Markgräfler Straße 16, 69126 Heidelberg
– im Folgenden „Provider“ genannt – und den vertraglich berechtigten Nutzern dieser Software – im Folgenden „Kunden“ genannt.

\subsection{Vertragsgegenstand}
\begin{enumerate}
\item Vertragsgegenstand ist die Software „pretix“ (nachfolgend kurz Software), die auf Servern im Einflussbereich des Providers bereitgestellt wird. Sie dient der Online-Organisation von Veranstaltungen (z.B. Fortbildungen oder Konferenzen), insbesondere zur Verwaltung der Anmeldungen und der Bezahlung (Ticketshop).
\item Mit diesem Vertrag wird dem Kunden die Nutzungsmöglichkeit für die Software über einen Internetzugang im Rahmen eines Application Service Providing (ASP) eingeräumt. Der Kunde darf die Software für eigene Zwecke nutzen, seine Daten verarbeiten und speichern.
\item Die Software, die für die Nutzung erforderliche Rechnerleistung sowie der notwendige Speicherplatz für Daten werden vom Provider oder einem von ihm beauftragten Dritten (z.B. einem Rechenzentrum) bereitgehalten. Der dem Kunden zugewiesene Systembereich ist gegen den Zugriff Dritter geschützt.
\item Der Zugang des Kunden zum Internet ist nicht Gegenstand dieses Vertragsverhältnisses. Der Kunde trägt die alleinige Verantwortung für die Funktionsfähigkeit seines Internet-Zugangs einschließlich der Übertragungswege sowie seines eigenen Computers.
\item Der Provider führt keine Aufgaben für den Kunden durch, insbesondere bearbeitet er nicht die über den Ticketshop ausgelösten Zahlungen.
\item Der Provider übermittelt dem Kunden die für die Softwarenutzung erforderlichen Zugangsdaten zur Identifikation und Authentifikation. Dem Kunden ist es nicht gestattet, diese Zugangsdaten Dritten zu überlassen.
\end{enumerate}

\subsection{Leistungen des Providers}
\begin{enumerate}
\item Auf dem Server werden die Inhalte unter der vom Kunden zur Verfügung zu stellenden Internet-Adresse zum Abruf über das Internet bereitgehalten. Die Leistungen des Providers bei der Übermittlung von Daten beschränken sich allein auf die Datenkommunikation zwischen dem vom Provider betriebenen Übergabepunkt des eigenen Datenkommunikationsnetzes an das Internet und dem für den Kunden bereitgestellten Server. Eine Einflussnahme auf den Datenverkehr außerhalb des eigenen Kommunikationsnetzes ist dem Provider nicht möglich. Eine erfolgreiche Weiterleitung von Informationen von oder zu dem die Inhalte abfragenden Rechner ist daher insoweit nicht geschuldet.
\item Der Provider erbringt die vorgenannten Leistungen mit einer Gesamtverfügbarkeit von 98,5 \% im Jahresschnitt. Die Einzelheiten zur Systemverfügbarkeit sind in den Service Level Agreements im Anhang geregelt. 
\item Der Provider sichert die Daten des Kunden in regelmäßigen Abständen. Die Abstände werden in den Service Level Agreements detaillierter beschrieben. Insbesondere trifft er Vorkehrungen gegen Datenverluste aufgrund von Computerabstürzen sowie gegen Zugriffe durch unbefugte Dritte. Die Sicherung erfolgt stets für den gesamten Serverinhalt und umfasst unter Umständen auch die Daten weiterer Kunden. Der Kunde hat daher keinen Anspruch auf Herausgabe eines der Sicherungsmedien. Im Notfall können aber gesicherte Inhalte auf den Server zurücküberspielt werden. Jenseits von Notfällen hat der Kunde keinen Anspruch auf Rückübertragung, kann sie aber mit dem Provider zu Tarifen gemäß Preisliste entgeltlich vereinbaren.
\item Der Provider ist berechtigt, die zur Erbringung der Leistungen eingesetzte Hard- und Software an den jeweiligen Stand der Technik anzupassen. Ergeben sich aufgrund einer solchen Anpassung zusätzliche Anforderungen an die vom Kunden auf dem Server abgelegten Inhalte, um das Erbringen der Leistungen des Providers zu gewährleisten, so wird der Provider dem Kunden diese zusätzlichen Anforderungen mitteilen. Der Kunde wird unverzüglich nach Zugang der Mitteilung darüber entscheiden, ob die zusätzlichen Anforderungen erfüllt werden sollen und bis wann dies geschehen wird. Erklärt der Kunde nicht bis spätestens vier Wochen vor dem Umstellungszeitpunkt, dass er seine Inhalte rechtzeitig zur Umstellung, dass heißt spätestens drei Werktage vor dem Umstellungszeitpunkt, an die zusätzlichen Anforderungen anpassen wird, hat der Provider das Recht, das Vertragsverhältnis mit Wirkung zum Umstellungszeitpunkt zu kündigen.
\end{enumerate}
\subsection{Serviceleistungen}
\begin{enumerate}
\item Die vom Provider zu erbringenden Serviceleistungen werden im Service Level Agreement (SLA) detailliert festgelegt. Das SLA wird als Anhang zu diesem Vertrag geführt und ist Teil der vertraglichen Übereinkunft.
\item Der Provider ist berechtigt, den Inhalt der Serviceleistungen einschließlich der bereitgestellten Software zu verändern und anzupassen, insbesondere bei technologischen Weiterentwicklungen. Bei Wegfall wesentlicher Funktionsbestandteile wird er den Kunden spätestens einen Monat vor der Änderung in Kenntnis setzen. In diesem Fall steht dem Kunden ein Sonderkündigungsrecht mit einer Frist von zwei Wochen zum Änderungstermin zu.
\item Nicht in (2) eingeschlossen sind Funktionen, die als „experimentell“, „beta“ o.ä. gekennzeichnet sind und vom provider ohne weitere Ankündigung abgeändert oder entfernt werden können.
\end{enumerate}

\subsection{Mitwirkungspflichten des Kunden}
\begin{enumerate}
\item Bei der Umschreibung, Eingrenzung, Feststellung und Meldung von Störungen muss der Kunde die vom Provider erteilten Hinweise befolgen. Gegebenenfalls muss der Kunde vom Provider zur Verfügung gestellte Checklisten zur Beschreibung verwenden.
\item Der Kunde muss seine Störungsmeldungen und Fragen nach Kräften präzisieren. Er muss hierfür gegebenenfalls auf kompetente Mitarbeiter zurückgreifen.
\item Der Kunde führt regelmäßige Datensicherungen durch und trifft geeignete Maßnahmen zur Sicherung seines eigenen Computers, z.B. den Einsatz eines Virenschutzprogramm in aktueller Version.
\item Der Kunde verhindert den unbefugten Zugriff Dritter auf die Software und verpflichtet auch seine Mitarbeiter zur Einhaltung dieser Pflicht.
\item Der Kunde verpflichtet sich, auf dem zur Verfügung gestellten Speicherplatz keine rechtswidrigen, die Gesetze, behördlichen Auflagen oder Rechte Dritter verletzenden Inhalte abzulegen. 
\item Der Kunde darf insbesondere nicht \begin{enumerate}
\item beleidigende oder verleumderische Inhalte, pornografische, gewaltverherrlichende, missbräuchliche, sittenwidrige oder Jugendschutzgesetze verletzende Inhalte, Waren oder Dienstleistungen bewerben, anbieten oder vertreiben;
\item andere Nutzer unzumutbar belästigen, insbesondere durch Spam (vgl. § 7 Gesetz gegen den unlauteren Wettbewerb – UWG);
\item gesetzlich (z. B. durch das Urheber-, Marken-, Patent-, Geschmacksmuster- oder Gebrauchsmusterrecht) geschützte Inhalte verwenden, ohne dazu berechtigt zu sein;
\item gesetzlich geschützten Waren oder Dienstleistungen, ebenfalls ohne dazu berechtigt zu sein, bewerben, anbieten oder vertreiben,
\item wettbewerbswidrige Handlungen, einschließlich progressiver Kundenwerbung (wie Ketten-, Schneeball- oder Pyramidensysteme) vornehmen oder fördern.
\end{enumerate}
\item Der Kunde ist verpflichtet, die folgenden belästigenden Handlungen zu unterlassen, auch wenn diese konkret keine Gesetze verletzen sollten: \begin{enumerate}
\item Versendung von Kettenbriefen;
\item Durchführung, Bewerbung und Förderung von Strukturvertriebsmaßnahmen (wie Multi-Level-Marketing oder Multi-Level-Network-Marketing); sowie Vornahme von anzüglicher oder sexuell geprägter Kommunikation (explizit oder implizit).
\item jede Handlung, die geeignet ist, die Funktionalität der pretix-Infrastruktur zu beeinträchtigen, insbesondere diese übermäßig zu belasten.
\end{enumerate}

\item Der Kunde ist dafür verantwortlich, dass die von ihm über das Angebot des Providers veröffentlichten Inhalte allen anwendbaren gesetzlichen Bestimmungen und behördlichen Auflagen entspricht. Hierzu zählen beispielhaft die Impressumspflicht, die Pflichten nach Art. 13 und 14 DSGVO, die Beachtung bestehender Urheberrechte, Hinweise auf eventuell bestehende AGB des Kunden, sowie (sofern sich das Angebot des Kunden an Verbraucher richtet) nach dem Verbraucherschutzrecht vorgeschriebene Hinweise wie z.B. auf eventuell bestehende Widerrufs- oder Rückgaberechte. Der Kunde wird weiterhin dafür Sorge tragen, dass die von ihm gewählte Internet-Adresse, unter der die Inhalte über das Internet abgefragt werden können, ebenfalls nicht Gesetze, behördliche Auflagen oder Rechte Dritter verletzt. Der Kunde stellt den Provider von jeglicher Inanspruchnahme durch Dritte einschließlich der durch die Inanspruchnahme ausgelösten Kosten frei.
\item Im Falle eines unmittelbar drohenden oder eingetretenen Verstoßes gegen die vorstehenden Verpflichtungen sowie bei der Geltendmachung nicht offensichtlich unbegründeter Ansprüche Dritter gegen den Provider auf Unterlassen der vollständigen oder teilweisen Darbietung der auf dem Server abgelegten Inhalte über das Internet ist der Provider berechtigt, unter Berücksichtigung auch der berechtigten Interessen des Kunden die Anbindung dieser Inhalte an das Internet ganz oder teilweise mit sofortiger Wirkung vorübergehend einzustellen. Der Provider wird den Kunden über diese Maßnahme unverzüglich informieren.
\item Die von dem Kunden auf dem Server abgelegten Inhalte können urheber- und datenschutzrechtlich geschützt sein. Der Kunde räumt dem Provider das Recht ein, die von ihm auf dem Server abgelegten Inhalte bei Abfragen über das Internet zugänglich machen zu dürfen, insbesondere sie hierzu zu vervielfältigen und zu übermitteln sowie sie zum Zwecke der Datensicherung vervielfältigen zu können. Der Kunde prüft in eigener Verantwortung, ob die Nutzung personenbezogener Daten durch ihn datenschutzrechtlichen Anforderungen genügt.
\item Der Kunde sichert zu, dass alle durch ihn angegebenen Daten der Wahrheit entsprechen, sowie dass er kein Verbraucher im Sinne des § 13 BGB ist. Diese Angaben sind auf Anfrage durch den Provider nachzuweisen. Bei Änderungen der Firmierung, Anschrift, Bankverbindung, Ansprechpartner oder Kontaktdaten informiert der Kunde den Provider unverzüglich, in der Regel durch selbstständige Änderung der Stammdaten im Kundenprofil.
\end{enumerate}


\subsection{Reseller-Ausschluss}

Der Kunde darf die vom Provider zur Verfügung gestellten Leistungen Dritten nicht zur gewerblichen Nutzung überlassen.

\subsection{Vergütung}
\begin{enumerate}
\item Der Kunde hat für die von ihm gewählten Service-Kategorien die sich aus der bei Vertragsschluss gültigen Preisliste des Providers ergebenden Entgelte zu zahlen.
\item Der Provider ist berechtigt, die seinen Leistungen zugrunde liegende Preisliste nach billigem Ermessen (§ 315 Abs. 3 BGB) zu ändern. Der Provider wird den Kunden über Änderungen in der Preisliste spätestens sechs Wochen vor Inkrafttreten der Änderungen in Textform informieren.
\item Das Entgelt wird nach Monatsabschnitten berechnet. Es ist am 1. Werktag des Monats fällig, der auf den Monat folgt, in dem die Einnahmen erzielt wurden.
\item Wenn keine andere Zahlungsweise schriftlich vereinbart wurde, wird der Kunde den Provider ermächtigen, die Miete im Lastschrifteinzugsverfahren einzuziehen, und für die erforderliche Deckung seines Bankkontos sorgen.
\end{enumerate}

\subsection{Vertragsschluss, Vertragslaufzeit und Kündigungen}
\begin{enumerate}
\item Die Software wird auf der Seite www.pretix.eu vorgestellt. Die Darstellung dort stellt kein rechtlich verbindendes Vertragsangebot dar. Der Vertrag kommt zustande, indem zunächst der Kunde auf der oben genannten Seite ein Angebot auf Abschluss des Vertrages abgibt und dann der Provider eine Erklärung abgibt, dass er den Antrag des Kunden auf Vertragsschluss annimmt. 
\item Die betriebsfähige Bereitstellung der vereinbarten Leistungen erfolgt unmittelbar nach der Mitteilung des Providers über den Vertragsschluss.
\item Der ASP-Vertrag läuft auf unbestimmte Zeit. Er kann zum Ende eines Kalender-Quartals schriftlich (z.B. per Brief, E-Mail) gekündigt werden.
\item Der Kunde kann den Vertrag jederzeit schriftlich (z.B. per Brief, E-Mail) kündigen.
\item Das Recht der Vertragsparteien zur Kündigung aus wichtigem Grund ohne Einhaltung einer Kündigungsfrist bleibt unberührt. Ein wichtiger Grund liegt insbesondere vor, wenn \begin{itemize}
    \item ein Vertragspartner die in diesem Vertrag ausdrücklich geregelten Pflichten grob verletzt,
    \item über das Vermögen der anderen Vertragspartei das Insolvenzverfahren eröffnet wird oder die andere Vertragspartei insolvent oder zahlungsunfähig wird
    \item der Kunde für zwei aufeinander folgende Termine mit der Entrichtung des Entgelts oder eines nicht unerheblichen Teils des Entgelts in Verzug ist, oder in einem Zeitraum, der sich über mehr als zwei Termine erstreckt, mit der Entrichtung des Entgelts in Höhe eines Betrags in Verzug ist, der das Entgelt für zwei Monate erreicht
    \item die Inhalte der vom Kunden über das System vertriebenen Veranstaltungen oder die auf anderen Kanälen vom Kunden publizierten Materialien rechtswidrige, sittenwidrige, volksverhetzende, rechtsradikale, gewaltverherrlichende oder sonstwie menschenverachtende Bestandteile enthalten,
    \item die Inhalte der vom Kunden über das System vertriebenen Veranstaltungen oder die auf anderen Kanälen vom Kunden publizierten Materialien dazu geeignet sind, den Ruf des Providers grob zu schädigen oder
    \item der Provider begründeten Anlass zur Vermutung hat, dass der Kunde in betrügerischer oder missbräuchlicher Absicht tätig ist.
\end{itemize}
    Für den Kunden kann ein wichtiger Grund in einer erheblichen Unterschreitung der vereinbarten Verfügbarkeit der Software liegen; hiervon ist regelmäßig bei einem Unterschreiten um mehr als 10\% auszugehen.
\end{enumerate}

\subsection{Recht des Providers zur Sperrung bei Zahlungsverzug des Kunden}
\begin{enumerate}
\item Der Kunde ist zur pünktlichen Zahlung des Entgelts verpflichtet. Bei einer Verzögerung von über zwei Wochen ist der Provider zur Sperrung des Zugangs berechtigt. Der Vergütungsanspruch bleibt von einer solchen Zugangssperrung unberührt. Die erneute Freischaltung erfolgt unmittelbar nach der Begleichung der Rückstände.
\item Die Verfolgung weitergehender Ansprüche, etwa nach dem Urheberrechtsgesetz, sowie insbesondere auch von sonstigen Schadensersatzansprüchen bleibt in allen Fällen vorbehalten.
\end{enumerate}

\subsection{Mängelansprüche und Kündigungsrecht des Kunden}
\begin{enumerate}
\item Mängel der Software einschließlich der Handbücher und sonstiger Unterlagen werden vom Provider nach entsprechender Mitteilung des Mangels durch den Kunden innerhalb der im SLA festgelegten Reaktionszeit behoben. Gleiches gilt für sonstige Störungen der Möglichkeit zur Softwarenutzung. Für die Mängelansprüche gilt mietvertragliches Mängelrecht.
\item Der Kunde darf eine Entgeltminderung nicht durch Abzug vom vereinbarten Entgelt durchsetzen. Entsprechende Bereicherungs- oder Schadensersatzansprüche bleiben unberührt.
\item Das Kündigungsrecht des Kunden wegen Nichtgewährung des Gebrauchs nach § 543 Absatz 2 Satz 1 Nr. 1 des Bürgerlichen Gesetzbuchs ist ausgeschlossen, sofern nicht die Herstellung des vertragsgemäßen Gebrauchs als fehlgeschlagen anzusehen ist.
\item Der Kunde hat dem Provider Mängel unverzüglich anzuzeigen. Die Mängelansprüche verjähren in einem Jahr.
\end{enumerate}

\subsection{Haftung}
\begin{enumerate}
\item Die Haftung des Providers für Schäden aufgrund der Nutzung von Telekommunikationsdienstleistungen für die Öffentlichkeit richtet sich nach den Regelungen des Telekommunikationsgesetzes.
\item Außerhalb des Anwendungsbereichs von Absatz (1) richtet sich die Haftung nach den folgenden Bestimmungen. Der Provider haftet unbeschränkt nur für Vorsatz und grobe Fahrlässigkeit auch seiner gesetzlichen Vertreter und leitenden Angestellten. Für das Verschulden sonstiger Erfüllungsgehilfen wird die Haftung auf das Fünffache des durchschnittlichen monatlichen Entgelts begrenzt.
\item Für leichte Fahrlässigkeit haftet der Provider nur bei Verletzung einer wesentlichen Vertragspflicht, deren Erfüllung die ordnungsgemäße Durchführung des Vertrags überhaupt erst ermöglicht und auf deren Einhaltung der Kunde regelmäßig vertrauen darf (Kardinalpflicht) sowie bei Schäden aus der Verletzung des Lebens, des Körpers oder der Gesundheit. Der Provider haftet dabei nur für vorhersehbare Schäden, mit deren Entstehung typischerweise gerechnet werden muss. Die Haftung ist im Falle leichter Fahrlässigkeit auf das Fünffache des durchschnittlichen monatlichen Entgelts begrenzt.
\item Die Haftung für Datenverlust wird auf den typischen Wiederherstellungsaufwand beschränkt, der bei regelmäßiger und gefahrentsprechender Anfertigung von Sicherungskopien eingetreten wäre. Für den Verlust von Daten und/oder Programmen haftet der Provider insoweit nicht, als der Schaden darauf beruht, dass es der Kunde unterlassen hat, Datensicherungen durchzuführen und dadurch sicherzustellen, dass verloren gegangene Daten mit vertretbarem Aufwand wiederhergestellt werden können.
\item Der Kunde erkennt an, dass eine 100\%ige Verfügbarkeit des Ticketshops technisch nicht zu erreichen ist. Insbesondere Wartungs-, Sicherheits- oder Kapazitätsbelange sowie Ereignisse, die nicht im Machtbereich des Providers stehen (wie z. B. Stromausfälle, Störungen von öffentlichen Kommunikationsnetzen etc.), können zu Störungen oder zur vorübergehenden Einstellung des Dienstes führen. Der Provider haftet gegenüber dem Kunden nicht für Beeinträchtigungen des Betriebes in Folge von Ereignissen, welche nicht in seinem Einflussbereich liegen, insbesondere höherer Gewalt und Dienstleistungen, welche durch Partnerunternehmen des Providers erbracht werden.
\item Der Provider schließt die Haftung für entgangene Umsätze des Kunden im Rahmen des oben genannten aus.
\end{enumerate}

\subsection{Datenschutz und Geheimhaltung}
\begin{enumerate}
\item Der Provider gewährleistet die datenschutzrechtliche Sicherheit der vom Kunden eingestellten Daten und beachtet die gesetzlichen Vorschriften zum Datenschutz, insbesondere das Telemediengesetz, das Telekommunikationsgesetz und das Bundesdatenschutzgesetz in der jeweils geltenden Fassung.
\item Der Provider unterrichtet hiermit den Kunden, personenbezogene Daten zu erheben, zu verarbeiten und zu nutzen, soweit dies für die Durchführung des ASP notwendig ist. Der Kunde ist damit einverstanden, dass seine Daten vom Provider gespeichert, übermittelt, gelöscht und gesperrt werden, soweit dies unter Abwägung der berechtigten Belange des Kunden und des Zwecks dieses Vertrags notwendig ist. Hierbei handelt es sich um eine Auftragsverarbeitung im Sinne von Art. 28 DSGVO, während der Kunde Verantwortlicher i.S.d. DSGVO bleibt. Auf Wunsch stellt der Provider einen gesonderten Vertrag zur Auftragsverarbeitung bereit.
\item Der Provider wird alle Informationen und Daten vertraulich behandeln, die ihm im Rahmen der Abwicklung dieses Vertragsverhältnisses vom Kunden zugänglich gemacht werden. Dies betrifft insbesondere Informationen über vom Kunden verwendete Methoden, Verfahren und Geschäftsgeheimnisse, Geschäftsverbindungen, Preise sowie Informationen über die Vertragspartner des Kunden. Der Provider ist ferner verpflichtet, den unbefugten Zugriff Dritter auf die Informationen und Daten des Kunden durch geeignete Vorkehrungen zu verhindern.
\item Der Provider ist verpflichtet, die Geheimhaltung gegenüber Dritten auch durch seine Mitarbeiter sicherzustellen.
\item Die Geheimhaltungspflicht gilt nach Vertragende noch drei weitere Jahre. 
\item Der Provider wird bei Vertragsende auf Wunsch des Kunden sämtliche Daten auf transportable Datenträger überspielen und dem Kunden aushändigen oder ihm online übermitteln. Der Kunde hat keinen Anspruch darauf, eine Software zur Verwendung der Daten zu erhalten. Nach einer Kontrolle der Daten durch den Kunden wird der Provider sämtliche Daten des Kunden löschen. Die Daten des Kunden bleiben in Sicherungskopien noch bis zu drei Monate nach der Löschung enthalten.
\end{enumerate}

\subsection{Haftpflichtversicherung}
Zur Sicherung etwaiger Ersatzansprüche des Kunden aus diesem Vertrag wurde vom Provider eine Haftpflichtversicherung in Höhe von 250.000,00 € abgeschlossen. Die Versicherungspolice wird dem Kunden auf Nachfrage nach Vertragsabschluss vorgelegt. Auf Verlangen des Kunden ist der Provider während der Vertragslaufzeit verpflichtet, den Nachweis laufender Beitragszahlungen an die Haftpflichtversicherung zu erbringen.

\subsection{Änderung der Vertragsbedingungen}
\begin{enumerate}
\item Soweit nicht bereits anderweitig speziell geregelt, ist der Provider berechtigt, diese Vertragsbedingungen wie folgt zu ändern oder zu ergänzen. Der Provider wird dem Kunden die Änderungen oder Ergänzungen spätestens sechs Wochen vor ihrem Wirksamwerden in Textform ankündigen. Ist der Kunde mit den Änderungen oder Ergänzungen der Vertragsbedingungen nicht einverstanden, so kann er den Änderungen mit einer Frist von einer Woche zum Zeitpunkt des beabsichtigten Wirksamwerdens der Änderungen oder Ergänzungen widersprechen. 
\item Der Widerspruch bedarf der Textform. Widerspricht der Kunde nicht, so gelten die Änderungen oder Ergänzungen der Vertragsbedingungen als von ihm genehmigt. Der Provider wird den Kunden mit der Mitteilung der Änderungen oder Ergänzungen der Vertragsbedingungen auf die vorgesehene Bedeutung seines Verhaltens besonders hinweisen.
\end{enumerate}

\subsection{Abtretung, Zurückbehaltungsrecht, Aufrechnung}
\begin{enumerate}
\item Die Abtretung von Forderungen ist nur mit vorheriger schriftlicher Zustimmung der anderen Vertragspartei zulässig. Die Zustimmung darf nicht unbillig verweigert werden. Die Regelung des § 354a HGB bleibt hiervon unberührt.
\item Ein Zurückbehaltungsrecht kann nur wegen Gegenansprüchen aus dem jeweiligen Vertragsverhältnis geltend gemacht werden.
\item Die Vertragsparteien können nur mit Forderungen aufrechnen, die rechtskräftig festgestellt oder unbestritten sind.
\end{enumerate}
\subsection{Schriftform}
Sämtliche Vereinbarungen, die eine Änderung, Ergänzung oder Konkretisierung dieser Vertragsbedingungen beinhalten, sowie besondere Zusicherungen und Abmachungen sind schriftlich niederzulegen. Werden sie von Vertretern oder Hilfspersonen des Providers erklärt, sind sie nur dann verbindlich, wenn der Provider hierfür seine schriftliche Zustimmung erteilt.

\subsection{Kollision mit anderen Geschäftsbedingungen}
Sofern der Kunde ebenfalls Allgemeine Geschäftsbedingungen verwendet, kommt der Vertrag auch ohne ausdrückliche Einigung über den Einbezug Allgemeiner Geschäftsbedingungen zustande. Soweit die verschiedenen Allgemeinen Geschäftsbedingungen inhaltlich übereinstimmen, gelten diese als vereinbart. An die Stelle sich widersprechender Einzelregelungen treten die Regelungen des dispositiven Rechts. Gleiches gilt für den Fall, dass die Geschäftsbedingungen des Kunden Regelungen enthalten, die im Rahmen der vorliegenden Geschäftsbedingungen nicht enthalten sind. Enthalten vorliegende Geschäftsbedingungen Regelungen, die in den Geschäftsbedingungen des Kunden nicht enthalten sind, so gelten die vorliegenden Geschäftsbedingungen.
\subsection{Salvatorische Klausel}
Sollten einzelne Bestimmungen der Parteivereinbarungen ganz oder teilweise unwirksam sein oder werden, wird die Wirksamkeit der übrigen Bestimmungen hierdurch nicht berührt. Die Parteien verpflichten sich für diesen Fall, die ungültige Bestimmung durch eine wirksame Bestimmung zu ersetzen, die dem wirtschaftlichen Zweck der ungültigen Bestimmung möglichst nahe kommt. Entsprechendes gilt für etwaige Lücken der Vereinbarungen.
\subsection{Rechtswahl}
Die Parteien vereinbaren hinsichtlich sämtlicher Rechtsbeziehungen aus diesem Vertragsverhältnis die Anwendung des Rechts der Bundesrepublik Deutschland.
\subsection{Gerichtsstand}
Für sämtliche Streitigkeiten, die im Rahmen der Abwicklung dieses Vertragsverhältnisses entstehen, wird Heidelberg als Gerichtsstand vereinbart.

\newpage
\section{Service Level Agreement und Technische Spezifikation}
\vspace*{-3mm}
\begin{center}– als Anlage zu den AGB –\end{center}
\vspace*{-3mm}

\subsection{Allgemeines}
\begin{enumerate}
\item Der Funktionsumfang der Software umfasst insbesondere \begin{itemize}
\item das Erstellen, Verwalten und Kategorisieren von Ticket-Produkten 
\item die technische Abwicklung eines Bestell- und Zahlungsvorgangs 
\item die Kontingentierung der Verfügbarkeit von Tickets 
\item den Zugang zum Verwaltungsportal für mehrere Nutzer 
\item eine Gutscheinfunktion für Werbeaktionen oder Ticket-Reservierungen 
\item den Export der Bestellerdaten in gängige Dateiformate.
\end{itemize}
\item Dem Kunden wird auf Wunsch die Möglichkeit eingeräumt, die Software vorab kostenfrei zu testen.
\item Die Software steht dem Kunden grundsätzlich rund um die Uhr zur Verfügung. Im Jahresmittel wird eine jährliche Mindestverfügbarkeit von 98,5 \% zugesagt. Höhere Gewalt oder Störungen außerhalb des Einflussbereichs des Providers sind hiervon ausgenommen. Wartungsarbeiten, die zu einem kurzfristigen Ausfall des Systems führen können, werden dem Kunden mindestens 48 Stunden vorher per E-Mail angekündigt. Ausgenommen hiervon sind solche Arbeiten, die zur Vermeidung oder Abwendung einer konkreten Gefahr (z.B. Sicherheitslücken) oder zur Behebung einer Störung kurzfristig notwendig sind. Für Funktionen, die mit „experimentell“, „beta“ o.ä. gekennzeichnet sind, gilt weiterhin keine Verfügbarkeitszusage.
\item Das System ist darauf ausgelegt, mindestens 100 Ticketverkäufe pro Minute störungsfrei abwickeln zu können. Bei besonders hoher Nachfrage einzelner Veranstaltungen kann auch der mögliche Durchsatz für andere Veranstalter kurzzeitig beeinträchtigt sein. Der Kunde ist aufgefordert, den Provider zwei Wochen vor Beginn des Ticketverkaufs zu kontaktieren, wenn mit einer sehr großen Nachfrage innerhalb weniger Minuten zu rechnen ist.
\item Alle für den regulären Betrieb notwendigen Systemkomponenten sind mindestens zweifach redundant innerhalb desselben Rechenzentrums ausgelegt, sodass der Ausfall eines einzelnen Servers keine direkten Auswirkungen auf die Erreichbarkeit hat.
\item Bei der Service-Plattform handelt es sich um Linux-Server aktueller Versionen.
\item Es werden geeignete, branchenübliche Maßnahmen getroffen, um die Sicherheit der Daten gegenüber unberechtigtem Zugriff sicherzustellen. Hierzu gehört, dass jede Kommunikation mit und zwischen unseren Servern nach aktuellen Standards verschlüsselt stattfindet. Es werden keine Passwörter im Klartext auf unseren Servern gespeichert.
\item Der Provider fertigt in der Regel alle vier Stunden, mindestens jedoch einmal täglich, Sicherungskopien der Daten an, die verschlüsselt in ein anderes Rechenzentrum übertragen und dort für mindestens eine Woche gespeichert werden. Die Wiederherstellung von Daten aus den Backups ist kostenlos, sofern der Anlass der Wiederherstellung durch den Provider verschuldet wurde. Andernfalls wird die Wiederherstellung nach Aufwand berechnet.
\item Der Provider strebt an, alle technischen Störungen innerhalb von 24 Stunden nach Eingang einer ausreichend aussagekräftigen Meldung zu beheben. Aufgrund der technischen Komplexität mancher Störungen kann dies jedoch nicht verbindlich zugesagt werden.
\end{enumerate}

\subsection{Organisation}
\begin{enumerate}
\item Erbringer des Services ist Raphael Michel, rami.io Softwareentwicklung.
\item Die Server sind angemietet und befinden sich in Rechenzentren der netcup GmbH in Nürnberg. Die Datensicherungen befinden sich auf angemieteten Servern des Rechenzentrums der Contabo GmbH in Nürnberg und München oder auf angemieteten Servern in den Rechenzentren der Hetzner Online GmbH in Nürnberg und Falkenstein. Diese Rechenzentren werden weiterhin als Fallback für das Produktivsystem im Falle eines länger andauernden Ausfalls der netcup GmbH verwendet.
\item Der Provider ist für den Kunden unter support@pretix.eu per E-Mail rund um die Uhr erreichbar. Der Provider strebt an, alle E-Mails schnellstmöglich zu beantworten, spätestens jedoch bis zum Abend des auf den Eingang der E-Mail folgenden Werktages.
\item Diese Frist gilt nicht für Supportanfragen von Kunden, die die Software lediglich im Rahmen des kostenlosen Nutzungsumfangs (siehe \ref{entgelt}) verwenden.
\item Der Provider stellt nach Möglichkeit telefonischen Support unter der auf der Website angegebenen Telefonnummer bereit. Die Erreichbarkeit per Telefon wird nicht garantiert. Der Kunde ist aufgefordert, im Falle eines Nichterreichens per Telefon seine Anfrage per E-Mail zu stellen oder eine Nachricht auf dem Anrufbeantworter zu hinterlassen. Dem Provider bleibt vorbehalten, telefonisch eingehende Anfragen per E-Mail an die vom Kunden hinterlegt E-Mail-Adresse zu beantworten.
\end{enumerate}

\subsection{Qualitätskontrolle}
\begin{enumerate}
\item Der Provider verwendet branchenübliche Monitoring-Software zur Messung der Verfügbarkeit. Hierbei wird die Erreichbarkeit der für den Betrieb relevanten Systeme regelmäßig getestet und im Falle eines Fehlers automatisch eine Nachricht an die technischen Mitarbeiter des Providers versendet. Die Monitoring-Infrastruktur befindet sich in einem anderen Rechenzentrum als die überwachten Systeme.
\item Die Messung der Erreichbarkeit findet in der Regel minütlich, mindestens jedoch alle 5 Minuten statt. Kurzfristige Unterbrechungen des Monitorings aufgrund technischer Wartung oder aufgrund von Systemausfällen sind möglich.
\item Die Ergebnisse dieser Messungen können auf Anfrage beim Provider eingesehen werden.
\end{enumerate}

\subsection{Entgeltberechnung}
\label{entgelt}
\begin{enumerate}
\item Für die Nutzung der Software fällt keine Grundgebühr an. Die Anzahl der Benutzer mit Zugriff auf die Software ist nicht begrenzt.
\item Für den Verkauf von Tickets oder anderen Produkten über die Software fällt eine Gebühr von 2,5 \% des erzielten Nettoumsatzes an, zuzüglich der gesetzlichen Umsatzsteuer.
\item Der Vertrieb kostenloser Tickets oder anderer kostenloser Produkte ist bis zu einer Gesamtmenge von 2.500 Verkaufspositionen pro Kalenderjahr kostenlos. Eine Gebühr für höhere Volumen wird zwischen Provider und Kunde individuell vereinbart.
\item Abweichende Entgeltvereinbarungen erfordern die Einhaltung der Textform.
\item Für die Erbringung zusätzlicher Dienstleistungen durch den Provider wird ein Stundensatz von 80 € netto vereinbart. Derartige Dienstleistungen werden nur aufgrund schriftlicher Beauftragung durch den Kunden durchgeführt. Auf Wunsch kann ein Kostenvoranschlag angefertigt werden.
\item Datensicherungen oder -exporte werden dem Kunden grundsätzlich auf digitalem Wege (z.B. per E-Mail oder Download) zur Verfügung gestellt. Wünscht der Kunde eine Übertragung auf einem physikalischen Datenträger, so wird der Aufwand berechnet. Die Bearbeitungsgebühr kann bis zu 120 € netto betragen.
\item Erhöhte Verfügbarkeits- oder Supportgarantien sind auf Anfrage gegen eine monatliche Grundgebühr möglich.
\end{enumerate}
\end{document}

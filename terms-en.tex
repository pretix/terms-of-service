\documentclass{terms}
\usepackage{multicol}
\usepackage{enumitem}
\usepackage[english]{babel}
\usepackage{scrextend}
\usepackage[utf8x]{inputenc}

\SetShortTitle{pretix - General T\&C (English translation, only German document is binding)}
\SetTitle{General Terms and Conditions\\ for contracts concerning the “pretix” ticket shop software}
\SetDate{2021-01-26}
\SetVersion{1.4}

\hypersetup{
    pdftitle={\Title},
    pdfauthor={pretix.eu},
    pdfsubject={},
    pdfkeywords={},
}
\setcounter{tocdepth}{1}

\begin{document}
\maketitle

\begin{center}
rami.io GmbH\\
Markgräfler Straße 16\\
69126 Heidelberg\\
Germany
\end{center}
\sloppy

\section*{Preamble}

\textbf{This is an English language translation of our General Terms and Conditions. We provide this translation for your information only. While we try to maintain a high accuracy of the translation, ONLY THE GERMAN VERSION OF THIS DOCUMENT IS LEGALLY BINDING.}
You can always obtain the German version on our website www.pretix.eu or contact us at the address stated above.

\section{General Terms and Conditions}
\subsection{General terms}
The Standard Business Terms apply to all contracts between rami.io GmbH, Markgräfler Straße 16, 69126 Heidelberg, Germany (referred to hereinafter as “Provider”) and the contractually-authorised user of the software (referred to hereinafter as “Customer”).

\subsection{Subject of the contract}
\begin{enumerate}
\item The subject of the contract is the software “pretix” (referred to hereinafter as “Software”) that is provided on servers within the Provider's sphere of control. The Software provides online organisational support for events (e.g. training courses or conferences), in particular managing registration and payment (“ticket shop”).
\item The purpose of this contract is to grant the Customer the right to use the Software via an Internet interface in the form of Application Service Providing (“ASP”). The Customer may use the Software for its own purposes and may process and store its own data.
\item The Software, computing power required for use and the storage space needed for data will be provided by the Provider or a third party engaged by the Provider (e.g. a computer center). The system area assigned to the Customer is protected against third-party access.
\item Customer access to the Internet is not included within the scope of this contract. The Customer is solely responsible for the functionality of its Internet access including transmission channels and its own computer.
\item The Provider performs no functions for the Customer, in particular the Provider does not process payments initiated via the ticket shop.
\item The Provider transmits identification and authentication access data required for the use of the Software to the Customer. The Customer is prohibited from providing such access data to third parties.
\end{enumerate}

\subsection{Services provided by the Provider}
\begin{enumerate}
\item Contents will be made available on the servers for access via the Internet at the URL to be provided by the Customer. Services to be performed by the Provider in relation to the transmission of data are limited solely to data communications between the transfer point operated by the Provider from its own data communications network to the Internet and the server provided for the Customer. The Provider has no ability to influence the flow of data outside of its own communications network. Accordingly, no obligation is owed to successfully transmit information from or to the computer accessing the content.
\item The Provider performs the services set out above with an average annual total availability of 98.5\%. The details related to system availability are set out in the attached Service Level Agreement. 
\item The Provider backs up Customer data at regular intervals. The intervals are described in more detail in the Service Level Agreement. In particular, it takes preventive measures against the loss of data related to computer failures as well as access by unauthorised third parties. In all cases, the back-up is performed for the complete contents of the server and may potentially include data from other customers. Accordingly, the Customer has no right to have back-up media provided to it. However, backed-up content may be re-loaded onto the server in the case of an emergency. Other than in cases of emergency, the Customer has no right to have data re-loaded, however it may agree to have this service performed by the Provider in exchange for a fee based on the price list.
\item The Provider is entitled to update hardware and software used to provide services to reflect the respective state-of-the-art. The Provider will notify the Customer of any additional requirements in the event such an update results in additional requirements for contents placed on the server by the Customer in order to ensure the provision of services by the Provider. Immediately following receipt of this notice, the Customer will decide whether such additional requirements will be satisfied and by what time this should be accomplished. If the Customer does not state that it will adapt its contents to the additional requirements on a timely basis (i.e. at the latest three days prior to the conversion date) four weeks before the planned conversion date at the latest, the Provider has the right to terminate the contract effective on the conversion date.

\end{enumerate}

\subsection{Specification of services}
\begin{enumerate}
\item The services to be provided by the Provider are set out in detail in the Service Level Agreement (“SLA”). The SLA is attached to this contract and is an integral component hereof.
\item The Provider is entitled to change or update the contents of services it provides including without limitation the Software it provides, in particular in the case of technological advancements. The Customer will be notified of a change at least one month in advance in the event a material functional components is discontinued. In such cases, the Customer has a special right of termination to be exercised at least two weeks prior to the date of the change.
\item Excluded from (2) is all functionality marked with labels like "experimental" or "beta". This functionality can be changed or removed by the Provider without advance notice.
\end{enumerate}

\subsection{Customer’s Duties of Co-Operation}
\begin{enumerate}
\item The Customer is required to follow instructions provided by the Provider in relation to the description, definition, determination and reporting of disruptions. If applicable, the Customer must use a check list from the Provider.
\item To the best of its ability, the Customer must make its error reports and questions concise. If applicable, it must involve qualified employees to do so.
\item The Customer will conduct data backups on a regular basis and will take appropriate action to ensure the security of its own computer, e.g. by installing the current version of an anti-virus software.
\item The Customer will prevent access to the Software by unauthorised third parties and will oblige its employees to comply with this obligation as well.
\item The Customer will not store any unlawful content or content that violates statutes, official regulations or the rights of third parties on storage space provided to it. 
\item In particular, the Customer may not \begin{enumerate}
\item Advertise, offer or distribute insulting or defamatory content, pornography, content, goods or services that glorify violence, are abusive, indecent or violate youth protection legislation;
\item Unreasonably harass other users, in particular Spam (cf. section 7 German Fair Trade Practices Act [UWG]);
\item Use legally protected content (e.g. laws related to copyrights, trademarks, industrial design or utility models) without authorisation;
\item Advertise, offer or distribute legally protected goods or services likewise without authorisation;
\item Undertake or promote anti-competitive acts including progressive customer advertising (e.g. chain, snowball or pyramid systems).
\end{enumerate}
\item The Customer is obliged to refrain from the following harassing acts even if they are not specifically unlawful:\begin{enumerate}
\item Sending chain letters;
\item Conducting, advertising or promoting any structural sales measures (e.g. multi-level marketing or multi-level network marketing) or undertaking any offensive or sexually charged communications (explicit or implicit).
\item Any act that is capable of impairing the functionality of the pretix infrastructure, in particular placing an excessive stain on it.
\end{enumerate}
\item The Customer shall ensure that all contents made accessibly by the Customer through the Provider's internet offering are in compliance with all applicable legal and regulatory requirements. This includes without limitation the obligation to make basic legal and contact information available ("imprint"), obligations under section 13 and 14 GDPR, obligations due to copyright law, correct notice about any Terms of Service of the customer, as well as (if applicable) required notices about the right of cancellation or return. The Customer shall ensure that the Internet address it selects at which the contents may be accessed online likewise does not violate laws, official regulations of the rights of third parties. The Customer shall indemnify and hold the Provider harmless from any third-party claims including any costs triggered by any such claims.
\item In the event of a breach of the foregoing obligations that is immanent or has already occurred, as well as the assertion of third-party claims against the Provider that are not obviously unfounded demanding the cessation of presenting contents stored on the server in whole or in part via the Internet, the Provider is entitled to temporarily suspend the connection of such contents to the Internet in whole or in part with immediate effect under consideration of the legitimate interests of the Customer. The Provider will inform the Customer immediately of such a measure.
\item Contents placed on the server by the Customer may be protected by copyright or data protection laws. The Customer grants the Provider the right to make contents it has placed on the service accessible via Internet query, in particular to duplicate and transmit such contents for this purpose as well as to make duplicates for data backup purposes. The Customer is responsible for reviewing whether its use of personal data satisfies data-protection related requirements.
\item The Customer ensures that all contact and business information provided by the Customer to the Provider is correct, as well as that the Customer is not a consumer according to the definition of § 13 German Civil Code (BGB). The Customer is obliged to prove the correctness of the submitted data on request by the Provider. If the company name, address, bank account, contact person or contact information of the Customer changes, the Customer will inform the Provider promptly, usually by changing the data in the customer profile on the Provider’s website.
\end{enumerate}
\subsection{Reseller exclusion}
The Customer may not provide the services provided by the Provider to a third party for commercial use.
\subsection{Fees}
\begin{enumerate}
\item The Customer is obliged to pay fees determined based on the service categories it has selected and the Provider's price list in effect upon conclusion of the contract.
\item The Provider is entitled to amend the price list for its services in the reasonable exercise of its discretion (section 315 para. 3 German Civil Code [BGB]). The Provider will provide the Customer written notice of changes in the price list at least six weeks before they take effect.
\item Fees are computed on a monthly basis. Payment is due on the first business day of the month following the month in which receipts were generated.
\item In the event a different means of payment has not been agreed in writing, the Customer will authorise the Provider to debit the rent by means of the direct debit process. The Customer is required to ensure adequate cover in its bank account.
\end{enumerate}
\subsection{Contract formation; Contract term and termination}
\begin{enumerate}
\item The Software is presented online at www.pretix.eu. This presentation does not comprise a legally binding offer to contract. The contract is formed when, first the Customer submits an offer to contract on the website listed above, and second the Provider then states that it accepts the Customer's offer to contract. 
\item Operational deployment of the agreed service follows directly after the Provider provides notice that the contract has been concluded.
\item The ASP contract is concluded for an indefinite term. It may be terminated in writing (e.g. letter or email) as of the end of a calendar quarter.
\item The Customer may terminate the contract at any time by providing written notice (e.g. letter or email).
\item The foregoing is without prejudice to the right of the parties to terminate the contract for good cause and with immediate effect. Good cause is satisfied if a party grossly violates a duty expressly provided in this contract and, in particular, \begin{itemize}
		\item in the event insolvency proceedings are opened in relation to the assets of the other party to the contract or the other party becomes insolvent or unable to make payments,
		\item if the Customer is in default of payment of fees for two successive periods or is in default with regard to a significant portion of the fee or is in default in a total amount equivalent to two months' fees over a period in excess of two periods,
		\item if the contents of the events the Customers sells and promotes through the systems or content published by the Customer on other channels contains illegal, immoral, race baiting, radical right-wing, violence-glorifying or otherwise inhuman elements,
		\item if the contents of the events the Customers sells and promotes through the systems or content published by the Customer on other channels are suitably to significantly damaging the Provider’s reputation, or
		\item the Provider has reason to believe the Customer is acting fraudulently or abusively.
	\end{itemize} From the standpoint of the Customer, good case may comprise a significant shortfall in the agreed level of availability for the Software. In general, this may be presumed in the event of a shortfall of more than 10\%.
\end{enumerate}
\subsection{Right of the Provider to suspend access in the event of default in payment by the Customer}
\begin{enumerate}
\item The Customer is obliged to pay fees on a timely basis. In the event of a delay of more than two weeks, the Provider is entitled to suspend access. Such a suspension of access does not affect the right to receive payment. Access will be restored immediately after any arrears has been brought current.
\item The right to assert additional claims, for example under the German Copyright Act [Urheberrechtsgesetz] or, without limitation, other claims for damages is reserved in all cases.
\end{enumerate}
\subsection{Claims for defects and the Customer's right of termination}
\begin{enumerate}
\item Defects in the Software, including manuals and other documents, will be cured by the Provider within the response time set out in the SLA following notice of the respective defect by the Customer. The foregoing applies in like manner with regard to disruptions in the ability to use the Software. The law applicable to defects in the context of rental contracts applies to claims for defects.
\item The Customer may not enforce a reduction in fees by means of applying a deduction to the agreed fees. This is without prejudice to corresponding claims to enrichment or compensation for damages.
\item The Customer's right of termination for failure to permit use pursuant to section 543 para. 2 sentence 1 no. 1 BGB is excluded unless the creation of contractual usage is to be viewed as having failed.
\item The Customer is required to notify the Provider of defects without delay. Claims for defects lapse after one year.
\end{enumerate}
\subsection{Liability}
\begin{enumerate}
\item Liability on the part of the Provider for damage due to the use of telecommunications services for the public is governed by the provisions of the Telecommunications Act [Telekommunikationsgesetz].
\item Other than as provided in paragraph (1), liability is governed by the following provisions: The Provider is liable without limitation only in cases of intent or gross negligence on its part or on the part of its representatives and executives. In the case of fault on the part of other agents, liability is limited to five times the average monthly fee.
\item In the case of simple negligence, the Provider is only liable for a breach of a material contract obligation, the satisfaction of which is a sine qua non of proper contract performance and the satisfaction of which may be regularly relied upon by the Customer (“cardinal obligation”), as well as damages resulting from injury to life, limb or health. In such cases, the Provider is only liable for foreseeable damages that may typically be expected in connection with the contract. In cases of simple negligence, liability is limited to five times the average monthly fee.
\item Liability for a loss of data is limited to the typical recovery expense which would have arisen had backup copies been prepared at regular intervals commensurate with risk levels. Accordingly, the Provider is not liable for the loss of data and/or programmes to the extent the damages result from the circumstance that the Customer did not prepare data backups so as to ensure that lost data could be recovered with a reasonable amount of expense and effort.
\item The Customer acknowledges that it is not technically possible to guarantee 100\% availability for the TicketShop. Without limitation, maintenance, security or capacity-related matters as well as events outside of the Provider's control (e.g. power outage, disruptions in the public communications network, etc.) may result in disruptions in or the temporary suspension of service. The Provider is not liable to the Customer for impairments to operations as a result of events that are outside of its control, including without limitation force majeure events and services that are provided by the Provider's partner companies.
\item Within the scope of the foregoing provisions, the Provider excludes liability for lost revenue on the part of the Customer.
\end{enumerate}
\subsection{Data protection and confidentiality}
\begin{enumerate}
\item The Provider guarantees security of data entered by the Customer in accordance with applicable data protection laws and complies with legal regulations related to data protection, in particular the provisions of the Telemedia Act [Telemediengesetz], the Telecommunications Act [Telekommunikationsgesetz] and the Federal Data Protection Act [Bundesdatenschutzgesetz] each as amended.
\item The Provider hereby informs the Customer that it collects, processes and uses personal data to the extent necessary for performance under the ASP. The Customer is in agreement that its data may be stored, transmitted, deleted and blocked by the Provider to the extent necessary under consideration of the Customer's legitimate interests and the purpose of this contract. The Provider is a data processor in terms of Art. 28 GDPR, while the Customer is the controller. On request, the Provider makes a special agreement on data processing available to the Customer.
\item The Provider will treat all information and data that is provided to it by the Customer within the scope of performance under this contract confidentially. Without limitation, this relates to information regarding methods and processes used by the Customer, business secrets, business relationships, prices as well as information regarding the Customer's contract partners. Furthermore, the Provider is obliged to take appropriate measures to prevent unauthorized access on the part of third parties to information and data of the Customer.
\item The Provider is obliged to ensure confidentiality in dealings with third parties on the part of its employees as well.
\item This non-disclosure obligation continues to apply for a period of three years following the termination of the contract. 
\item Upon the contract end date, the Provider will copy all data to portable data storage media at the Customer's request and either hand them over to the Customer or transmit such data to the Customer online. The Customer has no right to be provided software to use the data. After review of the data by the Customer, the Provider will delete all data belonging to the Customer. Back-up copies of Customer data will be retained for up to three months following deletion.
\end{enumerate}
\subsection{Liability insurance}
In order to secure any claims to compensation for damages on the part of the Customer under this contract, the Provider has concluded third-party liability insurance in the amount of € 250,000.00. Upon request, the insurance policy will be provided to the Customer after conclusion of the contract. Upon request of the Customer, the Provider is obliged to provide proof of payment of insurance premiums for liability insurance during the term of the contract.
\subsection{Amendments to contract terms}
\begin{enumerate}
\item To the extent not otherwise specifically agreed, the Provider is entitled to amend or supplement these contractual terms as set out below: The Provider will provide the Customer written notice of the changes or additions six weeks before they are to take effect. If the Customer does not agree to the changes or additions to the contractual terms, it may object to such changes up to one week prior to the planned date on which the respective changes or additions are to take effect. 
\item An objection must be in writing. If the Customer does not submit an objection, the amendments or additions to the contractual terms are deemed to have been approved by the Customer. When providing notice of an amendment or addition to the contractual terms, the Provider will specifically inform the Customer of the intended meaning of its behaviour.
\end{enumerate}
\subsection{Assignment; Right of retention; Offset}
\begin{enumerate}
\item Receivables may only be assigned with the prior written consent of the other party to the contract. Consent may only be refused for good cause. This is without prejudice to the provisions of section 354a German Commercial Code [HGB].
\item A right of retention may only be exercised in relation to counter-claims under the same contractual relationship.
\item The parties may only exercise a right of set-off where the respective claim is not disputed or has been finally determined by a court.
\end{enumerate}
\subsection{Written form}
All agreements containing an amendment, addition or specification of these contractual terms, as well as special assurances and understandings, must be in writing. If they are expressed by the representatives or agents of the Provider, they are only deemed to be binding if approved in writing by the Provider.
\subsection{Conflicts with other business terms}
In the event the Customer likewise uses standard business terms, the contract is nonetheless concluded even without express agreement on the inclusion of standard business terms. To the extent the content of different standard business terms coincides, they are deemed to have been agreed. Individual provisions that are in conflict shall be replaced by provisions permitted under dispositive law. The foregoing also applies in the event the Customer's standard business terms contain provisions that are not contained in these Standard Business Terms. If these Standard Business Terms contain provisions that are not included in the Customer's standard business terms, these Standard Business Terms control.
\subsection{Severability clause}
If specific provisions of agreements in place between the parties are or become invalid in whole or in part, the validity of the remaining provisions shall by unaffected thereby. In such cases, the parties undertake to replace the invalid provision with a valid provision that comes as close as possible to the business purpose of any such invalid provision. The foregoing applies in like manner to any loopholes in agreements between the parties.
\subsection{Choice of law}
The Parties agree to the application of the laws of the Federal Republic of Germany with regard to all legal relationships arising under this contract.
\subsection{Jurisdiction}
Heidelberg is agreed as the place of jurisdiction for all disputes arising in connection with the performance of this contract.

\newpage
\section{Service Level Agreement and technical specifications}
\begin{center}
- Annex to the General Terms and Conditions -
\end{center}
\subsection{General matters}
\begin{enumerate}
\item The functionality of the Software includes the following in particular \begin{itemize}
\item The creation, administration and categorization of ticket products 
\item The technical processing of an order and payment transaction 
\item The allocation of ticket availability 
\item Access to the administration portal for multiple users 
\item A voucher system for promotional activities or ticket reservations 
\item The export of customer data in common file formats
\end{itemize}
\item If desired, the Customer will be provided the opportunity to test the Software in advance at no cost.
\item The Software is fundamentally available to the Customer around the clock. Annual minimum availability of 98.5\% is assured on an annual average. The foregoing does not include force majeure events and disruptions outside of the control of the Provider. The Customer will be notified of maintenance work that could cause temporary unavailability of the system by email at least 48 hours in advance. The foregoing does not apply to work that is necessary on short notice to prevent or avoid a specific danger (e.g security gaps) or to resolve a malfunction. Features marked with labels like “beta” or “experimental” are not included in this assertion of availability.
\item The system is designed to be able to process at least 100 ticket sales per minutes without interruption. In the case of exceptionally high demand for specific events the potential throughput for other event organizers may also be affected temporarily. The Customer is requested to contact the Provider two weeks before the start of ticket sales if it expects very high demand within a span of a few minutes.
\item All system components required for normal operations are installed at the same computer center on at least a double-redundant basis so that failure of a single server has no direct effect on availability.
\item The service platform is made up of current generation Linux servers.
\item Appropriate, industry standard measures are implemented in order to secure data from unauthorized access. Such measures include the encryption of every communication with and between our servers based on current standards. No passwords are stored on our servers in plain text form.
\item As a rule, the Provider prepares backup copies every four hours, however once daily at the least, of data that are sent to a different computer center on an encrypted basis and are stored there for at least a week. Recovery of data from these back-ups is free of charge to the extent the Provider is at fault for the incident creating a need to recover data. In all other cases, recovery is charged based on cost.
\item The Provider will attempt to resolve all technical malfunctions within 24 hours of receipt of a sufficiently-detailed report. However, this cannot be provided as a binding commitment in light of the technical complexity of some malfunctions.
\end{enumerate}
\subsection{Organizational information}
\begin{enumerate}
\item The provider of the service is Raphael Michel, rami.io Softwareentwicklung.
\item The servers are leased and are located in computer centers provided by netcup GmbH in Nuremberg. The data backups are located on leased servers in computer centers provided by Contabo GmbH in Nuremberg and Munich or in computer centers provided by Hetzner Online GmbH in Nuremberg and Falkenstein. These computer centers will also be used as a fallback for the main system in case netcup GmbH experiences a longer downtime.
\item The Provider may be reached around the clock by email at support@pretix.eu. The Provider attempts to all respond to all emails as quickly as possible, at the latest however by the evening of the business day following receipt of the email.
\item The time limit set in (4) does not apply if the Customer only uses the Software as part of the free usage tier (see \ref{entgelt}).
\item Subject to availability, the Provider provides telephone support at the telephone number indicated on the website. Availability by telephone is not guaranteed. The Customer is requested to submit its inquiry by email if not able to reach the Provider by telephone or to leave a message on the answering machine. The Provider reserves the right to respond by email to inquiries received by telephone using the email address on file.
\end{enumerate}

\subsection{Quality control}
\begin{enumerate}
\item The Provider uses industry standard monitoring software to measure availability. In this context, availability for systems relevant for operations are test regularly and, in the event of an error, a message is sent automatically to the technical employees at the Provider. The monitoring infrastructure is located at a different computer center than the systems subject to monitoring.
\item As rule, availability is measured every minute, however at least every five minutes. Temporary interruptions in monitoring are possible as a result of technical maintenance or system failures.
\item Upon request, the results of monitoring may be reviewed at the Provider.
\end{enumerate}

\subsection{Calculation of fees}
\label{entgelt}
\begin{enumerate}
\item There is no basic fee for use of the Software. There is no limit on the number of users with access to the Software.
\item A fee in the amount of 2.5\% of the net price, plus statutory VAT, will charged for the sale of tickets or other products via the Software.
\item Usage of the Software to distribute free tickets or other free products is free of charge up to an amount of 2,500 order positions per calendar year. If the Customer plans to exceed this limit, the Provider and the Customer need to agree on an individual fee.
\item Fee agreements that vary from this must be in writing.
\item An hourly rate of € 80.00 is agreed for the provision of additional services by the Provider. Any such services will only be provided on the basis of a written order from the Customer. A cost estimate may be prepared on request.
\item Data backups or exports will fundamentally be provided to the Customer by electronic means (e.g. by email or download). A cost-based fee will be charged if the Customer wishes to receive a physical data carrier. The service charge may be up to € 120.00 (net).
\item Higher availability or support guarantees are possible on request in exchange for a monthly basic fee.
\end{enumerate}

\end{document}
